\section{implementaion of solution with Backtracking}
One of the good method for solving Sudoku is using Backtracking.
Sudoku can be solved by one by one assigning 
numbers to empty cells. Before assigning a 
number, check whether it is safe to assign. 
Check that the same number is not present in 
the current row, current column and current $3*3$
subgrid. After checking for safety, assign the 
number, and recursively check whether this assignment 
leads to a solution or not. If the assignment 
doesn’t lead to a solution, then try the next 
number for the current empty cell. And if none 
of the number (1 to 9) leads to a solution, 
return false and print no solution exists.

\subsection{implementaion in C++}
Our main fucntion is shown in the Listing \ref{list:sudoku}.
In the for loop all possibele choices for $k_{th}$ element in the
array of empty cells(emptycells is global array that is initialize with empty cells index and value of 0) are consirdered and if the choice is promising
we continue our process on the next element of empty cells array.
\begin{lstlisting}[style=mystyle,
language=C++, 
caption=main recursive function for implementing backtracking,
label=list:sudoku]
void sudoku(int k){
   for(int i=1; i<= N;++i){
       //The value of cell is char so we add '0' to assign ASCII number
       emptycells[k].v = '0' + i;
       if (promising(k)){
            if(k== emptycells.size()-1){ 
                print_table();
            }
            else
                sudoku(k+1);
       } 
   } 
}
\end{lstlisting}

Now lets see how the promising fucntion works. The code is
shown in the Listing \ref{list:promising}. The function has
four things to check the compatiblity.
\begin{enumerate}
  \item elements in the row
  \item elements in the column
  \item elements in the $3*3$ square
  \item previous choices of cells 0,...,k-1 of emptycells array.
\end{enumerate}

\begin{lstlisting}[style=mystyle,
language=C++, 
caption=promising function,
label=list:promising]
bool promising( int k){
    if (check_row(k) && check_col(k) && check_sq(k) && check_pre_cells(k))
        return true;
    else
        return false;
}
\end{lstlisting}

Implementation of each check functions are shown in Listing \ref{list:check}

\begin{lstlisting}[style=mystyle,
language=C++, 
caption=checking functions,
label=list:check]
//check if all elements of the row element k is in compatiblity with the choice of k
bool check_row(int k){
    //take care of constant valued cells
    for(int j=0; j<N; ++j){
        //to make sure is not the same cell
        if(emptycells[k].y != '0' +j){
            if (table[emptycells[k].x-'0'][j] == emptycells[k].v){
                return false;
            }
        }
    }
    return true;
}
bool check_col(int k){
    //take care of constant valued cells
    for(int i=0; i<N; ++i){
        // check it isn't the same element
        if(emptycells[k].x != '0'+i){
            if (table[i][emptycells[k].y-'0'] == emptycells[k].v){
                return false;
            }
        }
    }
    return true;
}
bool check_sq(int k){
    //finding the top left square in 3*3 square k is in 
    int sq_row = int(emptycells[k].x /3)*3;
    int sq_col = int(emptycells[k].x /3)*3;
    for(int i=sq_row; i <sq_row+3;++i){
        for(int j=sq_col ; j<sq_col+3;++j){
            if(emptycells[k].x -'0' != i || emptycells[k].y -'0' !=j)
            {
                if(emptycells[k].v == table[i][j])
                    return false;
            }
        }
    }
    return true;

}
//checking if the previous choices are compatible
bool check_pre_cells(int k){
    //the squre that element k is in 
    int sq_row = int(emptycells[k].x /3)*3;
    int sq_col = int(emptycells[k].y /3)*3;

    for(int i=0; i<k; ++i){
        //row_check
        if(emptycells[i].x == emptycells[k].x)
            if (emptycells[i].v == emptycells[k].v)
                return false;
        //col_check
        if(emptycells[i].y == emptycells[k].y)
            if (emptycells[i].v == emptycells[k].v)
                return false;
        //square_check
        if(int(emptycells[i].x/3)*3 == sq_row && int(emptycells[i].y/3)*3 == sq_col)
            if (emptycells[i].v == emptycells[k].v)
                return false;
    }
    return true;
}

\end{lstlisting}

Now to make our life easier for testing our functions on
different senarios of the puzzle we write codes to read puzzles
from a file that user has specified in the terminal argument
and after that store it in a 2-D array. Then we create
a vector of data structure cell that that has value x(row number),
y(column number) and v(value of the cell) to store the empty
cells that the program is meant to fill. You can see all of this
in Listing \ref{list:main}
\begin{lstlisting}[style=mystyle,
language=C++, 
caption=other parts of program,
label=list:main]
struct cell{
    char x;
    char y;
    char v;
};

//global variables
char table[N][N]; // the main sudoku table
std::vector<cell> emptycells;// list of empty cells

int main(int argc, char** argv){
    //getting file name from argv in terminal
    //else default value of game1.txt
    string filename;
    if (argc >1)
        filename = argv[1];
    else
       filename = "game1.txt";
    ifstream myfile;
    myfile.open(filename);
    string line;
    int i=0;
    while (getline(myfile,line)) {
        istringstream iss(line);
        for(int j=0; j<N; ++j){
            iss>>table[i][j];
            if (table[i][j] == '0'){
                cell c{char('0'+i),char('0'+j),'0'};
                emptycells.push_back(c);
            }
        }
        i++;     
       }
    sudoku(0);
    return 0;
}
\end{lstlisting}
Now with the help of g++ we can compile our program.
\begin{lstlisting}
$g++ -std=c++14 main.cpp -o out
\end{lstlisting}
now we store a puzzle in a file like the one is shown in
the Figure \ref{fig:file}
\begin{fig-shaded}{input.txt}{fig:file}
\begin{lstlisting}
4 0 0 0 0 2 8 3 0
0 8 0 1 0 4 0 0 2
7 0 6 0 8 0 5 0 0
1 0 0 0 0 7 0 5 0
2 7 0 5 0 0 0 1 9
0 3 0 9 4 0 0 0 6
0 0 8 0 9 0 7 0 5
3 0 0 8 0 6 0 9 0
0 4 2 7 0 0 0 0 3
\end{lstlisting}
\end{fig-shaded}
And then with following command you can solve the puzzle.
\begin{lstlisting}
$./out input.txt
\end{lstlisting}
This is what you get in the terminal:
\begin{fig-shaded}{input.txt}{fig:file}
\begin{lstlisting}
4,9,1,6,5,2,8,3,7,
5,8,3,1,7,4,9,6,2,
7,2,6,3,8,9,5,4,1,
1,6,9,2,3,7,4,5,8,
2,7,4,5,6,8,3,1,9,
8,3,5,9,4,1,2,7,6,
6,1,8,4,9,3,7,2,5,
3,5,7,8,2,6,1,9,4,
9,4,2,7,1,5,6,8,3,
\end{lstlisting}
\end{fig-shaded}